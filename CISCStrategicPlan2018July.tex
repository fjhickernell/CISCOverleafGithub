\documentclass[12pt]{amsart}

\usepackage{booktabs, hyperref}
\usepackage[dvipsnames]{xcolor}


\textwidth 6.5 in
\textheight 9 in
\hoffset -0.8 in
\voffset -0.5 in

\newcommand{\FJHNote}[1]{{\textcolor{blue}{FJH: #1}}}
\newcommand{\DMNote}[1]{{\textcolor{green}{DM: #1}}}
\newcommand{\MCNote}[1]{{\textcolor{orange}{MC: #1}}}
\newcommand{\SCCNote}[1]{{\textcolor{magenta}{SCC: #1}}}


\begin{document}
\FJHNote{This is Fred's note}, \MCNote{This is Maggie's note}, and \DMNote{This is David's note}.

\title[Center for Interdisciplinary Scientific Computation Strategic Plan]{Center for Interdisciplinary Scientific Computation \\ Strategic Plan}
\date{August 1, 2018}
\maketitle


To fulfill Illinois Tech's mission, ``To provide distinctive and relevant education in an environment of scientific, technological, and professional knowledge creation and innovation," we must strengthen and leverage our expertise in computation as tool for discovery.  The Center for Interdisciplinary Scientific Computation (CISC) is poised to play a key role. Here we summarize our history, purpose, past and future activities, and proposed funding sources.

Computation, alongside theory and experiment, is a crucial element in explaining phenomena, predicting what has not yet been observed, and optimizing performance.  Today, scientific computation spans the spectrum from simulation science to data science. Scientific computation draws upon multiple disciplines, including
\begin{itemize}
    \item Science and engineering domain knowledge to ensure that the problems posed are relevant and that the answers computed are significant,
    \item Computer science for novel algorithms, computational software, and hardware architectures that facilitate efficient and reliable computation, and
    \item Mathematical sciences for the framework justifying computational procedures and making inferences from data.
\end{itemize}
Thus, at Illinois Tech scientific computation cannot be owned or led by a single department, but is fostered through an interdisciplinary center.

\subsection*{Background.} Illinois Tech's expertise in scientific computation lies in all College of Science Departments as well as in other Illinois Tech colleges, schools and departments.  Following a long term discussion, CISC was established in May of 2017 to leverage this expertise for greater impact.  Fred Hickernell, Professor of Applied Mathematics was appointed as director.  Professor Hickernell's research in computational mathematics and statistics has been funded by the National Science Foundation and the Department of Energy.  In the spring of 2018, David Minh, Assistant Professor of Chemistry, was appointed Associate Director.  Professor Minh's research in \FJHNote{David, please fill in} ??? has been funded by ???.  The CISC Advisory Board is comprised of 
\begin{itemize}
    \item Professor Grant Bunker, Chair of the Department of Physics, 
    \item Professor John Georgiadis, Chair of  the Department and R. A. Pritzker Professor of Biomedical Engineering,
    
    \item Professor Ron Landis, Deputy Vice Provost for Research and Academic Affairs and
Nambury S. Raju Professor of Psychology,

    \item Professor Chun Liu, Chair of the Department of Applied Mathematics, and

    \item Professor Xian-He Sun, Distinguished Professor of Computer Science.
\end{itemize}

\subsection*{Vision.}
CISC will be a national and international center of excellence in scientific computation underpinning and catalyzing multiple research and educational activities at Illinois Tech within the university, in Chicago, and beyond.

\subsection*{Mission.}
To intensify computationally-driven scholarship and education, across the College of Science, Illinois Tech as a whole, and in greater Chicagoland.  This may lead to major scientific advances not otherwise possible.

\subsection*{Goals.}
To fulfill its mission, CISC initiates and promotes programs to enhance research, education, and community engagement.  Our goals are the following:
\begin{itemize}
    \item Attract substantial external funding to support major \emph{research},
    
    \item Develop comprehensive and multi-pronged scientific computation \emph{education},

    \item Strengthen Illinois Tech's research computing \emph{infrastructure}, and
    
    \item Engage the \emph{community}.
    
\end{itemize}
We expect to see substantial progress within a three-year time horizon.

\FJHNote{David, please fill in observations from our survey in the paragraphs below as appropriate.}

\subsection*{Research.} We expect to have at least a handful of externally funded major interdisciplinary scientific computation research programs at any time, with the particular themes depending our expertise and on the priorities of funding agencies.  A main weakness in competing for external funding for scientific computation research is our ignorance of what others in our own institution are doing and our lack of experience discussing problems across disciplinary boundaries.  CISC has and will take measures to overcome this weakness.

To promote interdisciplinary collaboration on computational projects, we instituted a series of \emph{lunchtime matchmaking seminars} in the fall of 2017.  Each seminar featured two professors highlighting their computational research and the areas for potential collaboration.  Four applied mathematicians, two biologists, two biomedical engineers, a business faculty, a chemical and biological engineers, two chemists, two computer scientists,  a mechanical engineer, a physicist, and a psychologist spoke during the 2017--2018 academic year.  

While continuing this series, we also want to facilitate the next step toward externally funded research by offering more substantial support to kickstart interdisciplinary collaborations.  We propose to offer \emph{summer research student stipends} to students supervised by faculty from more than one department.  These would be awarded based on the prospects of the summer work leading successful external funding of new interdisciplinary computational science research. 

In the fall of 2017, we invited interdisciplinary \emph{seed grant} proposals to fund research in the 2018 calendar year leading to new external funding proposals in scientific computation.  Members of the Advisory Board reviewed the proposals and awarded a \$40K seed grant to Lulu Kang, Sonja Petrovi\'c, and Mahima Saxena. The funds were earmarked for teaching relief, data collection, and conference travel.  We plan to continue this seed grant competition on an annual basis.  Applicants are given latitude in proposing how use the funds.  For example, seed grants might be used for partial post-doctoral scholar support.  The primary criterion for selection is the greatest prospect of future external funding of new initiatives.

CISC has sponsored a few \emph{guest lectures} given by scholars leading major large scale computation projects.  Our aim is to open the eyes of colleagues and students to what might be done.  Two of these guest lecturers were from the nearby \emph{Department of Energy labs}.  CISC hopes to strengthen our collaboration with these labs, which have access to advanced hardware and can provide valuable experience to our students. 

\subsection*{Education}Our scientific computation students are typically educated within their disciplinary silos, i.e., they are educated as computational biologists, computational chemists, computational mathematicians, computational physicists, or computer scientists.  Moreover, many students only gain experience with single CPU computation. We want to provide a richer experience for our students.

In early 2018, CISC submitted an NSF \emph{training grant} proposal entitled \emph{Cross-Disciplinary Education for Next-Generation Computational Scientists} to the Office of Advanced Cyberinfrastructure. The (Co-)PIs are Sou-Cheng Choi (Allstate), Hickernell, Minh, Sun, and Jeff Wereszczynski (physics), and Norm Lederman (science education) is a senior personnel.  A theme of this proposal is breaking down silos.  If funded, CISC will 
\begin{itemize}
    \item Place scientific computation students in research groups outside their disciplines so that they may learn how to think in new ways and work in interdisciplinary teams, some of these might be at government labs;
    
    \item Develop new scientific computation courses offered in various departments that can be taken by students from multiple disciplines, these courses will stress the use of high performance architectures, good professional practices of developing software for reproducible research, and scientific applications;
    
    \item Promote graduate theses incorporating advanced cyberinfrastructure; and
    
    \item Offer scientific computation research experiences for community college students and high school students.

\end{itemize}
The proposal---under review---includes partnerships with Argonne, College of DuPage, Fermilab, and Illinois Tech's Admissions Office.

We plan to continue to apply for training grants to support our aim of building a strong interdisciplinary scientific computation education for high school through PhD students.  Possible sources of funding include \FJHNote{Maggie help us out here.}

\subsection*{Infrastructure.} The Office of Technology Services (OTS) has turned its attention to research computing in the recent couple of years.  Illinois Tech has established gridIIT, our part of the Open Science Grid (\href{http://opensciencegrid.org}{\nolinkurl{opensciencegrid.org}}).  The Office of Technology Services has also been searching for ways to accommodate the computer clusters acquired through external grants.  

CISC intends to partner with OTS and the Illinois Tech administration to support scientific computation research in ways that individual faculty or departments cannot do on their own.  We are compiling a database of licenses held for major software packages used in scientific computation (\href{http://bit.ly/2LrQC4T}{\nolinkurl{bit.ly/2LrQC4T}}),  such as ABAQUS, ADF, and Gaussian.  The 256 node von Neumann cluster, purchased by the College of Science and managed by CISC, is on gridIIT and available for use campus wide.  Many scientific computation researchers would like to take advantage of available high performance computing resources, but the learning curve is steep.  We are in dialogue with OTS on how to flatten the curve by making templates and other training materials available for using gridIIT.  Wereszczynski is the campus champion for XSEDE (\href{https://www.xsede.org}{\nolinkurl{www.xsede.org}}), which we would like to see more of our faculty taking advantage of.

Community outreach

\subsection*{Membership.} 
Up to now CISC has had no formal membership.  We will begin inviting faculty to join as \emph{CISC Affiliates}, starting with those who have participated in past CISC activities.  CISC Affiliates will be asked to indicate their experience in scientific computation and how they plan to be engaged in CISC, e.g. submitting grants associated with CISC, offering lectures or tutorials, or participating in curriculum development. The number of potential CISC Affiliates, i.e., faculty involved in scientific computation, is several dozen.  

\subsection*{Budget.}
The initial CISC annual budget required to support our proposed activity is 
\begin{center}
    \begin{tabular}{p{0.57\textwidth}@{\qquad} r}
\textbf{Item} & \textbf{Annual Expense in \$K} \\
\toprule
    Seed grant  & 40 \\
    Summer research student stipends 3 @ \$7K each & 21 \\
    Cluster hardware upgrades & 30 \\
    Lunchtime matchmaking seminars & 2 \\
    Special lectures & 2 \\
    Student helper 30 weeks $\times$ 5 hours/week $\times$ \$15/hour & 7 \\
    Office expenses and other miscellany & 5 \\
    \bottomrule
    \emph{Total} & 107
\end{tabular}
\end{center}
This does not include teaching relief and administrative increment for the Director and Associate Director, which are presently supported by the College of Science.  As additional funds become available we would like to add 
\begin{center}
    \begin{tabular}{p{0.57\textwidth}@{\qquad} r}
\textbf{Item} & \textbf{Annual Expense in \$K} \\
\toprule
    Additional seed grant  & 40 \\
    Conference sponsorship & 10 \\
    \bottomrule
    \emph{Total} & 50
\end{tabular}
\end{center}

\subsection*{Financial Resources.}
We aim to have stable income of at least \$160K per year within three years.
There are several potential sources of income to support CISC.  

One is an \emph{endowment}.  We were under the impression that donors with substantial means were willing to support CISC until spring 2017, when it became clear otherwise.  We are eager to work with Institutional Advancement to approach potential donors to build an endowment.  We would also seek an endowed professorship for the next CISC director, also within three years.

Another potential source of income is the \emph{university's recurring budget}.  The College of Science supported CISC at a level of \$50K in the 2017-18 fiscal year, which was used to support the \$40K seed grant and various other CISC activities.

As highlighted under our goals, CISC is eager to partner with \emph{research} groups seeking external funding for major scientific computation projects.  We believe that CISC participation, with its infrastructure, will strengthen research proposals.  However, CISC's research scope is too broad for its activities in support of research to be externally funded.  We will compete for external \emph{training and education} grants, but these will not support much of the annual expenses outlined above.

We propose that a portion of the \emph{indirect costs} associated with all grants submitted in association with CISC be returned to CISC to support its annual expenses.  The advantages of this proposal are:
\begin{itemize}
    \item CISC is incentivized to support activities that lead to successful external proposals.
    
    \item Principal investigators (PIs) are incentivized to affiliate their proposals with CISC, knowing if successful, a portion of the indirect costs will be available to prime the pump for future proposals.  This assumes that the portion of indirect costs going to CISC does not affect the portion returned to the PI.
    
    \item The more successful CISC and our faculty are, the less CISC will rely on the recurrent university budget, and the more indirect costs will flow to the university.
\end{itemize}
This model is new to Illinois Tech and requires effort to be implemented.


\subsection*{Long Term Plans.}
The CISC name and this strategic plan mention ``scientific computation'' and omit ``engineering''.  CISC was founded and initially resourced out of the College of Science.  However, from the beginning we have not limited our scope to science only, but have welcomed engineering and other disciplines.  Dean of Science Russell Betts appointed two members of our Advisory Board from outside science in consultation with their deans.  Faculty and students outside of science have participated in our activities.

Our plan is that within five years CISC to grow into or become part of a university-wide institute that would promote computational science, engineering, human sciences, and business.  This proposed institute would engage multiple disciplines.  We hope that Provost Kilpatrick will work with the deans and us to make that a reality.  Illinois Tech should be known as a leader in computation.








\end{document}

Background
Modern science has been profoundly influenced by the development and use of computational methods for research and discovery.  This includes both large-scale simulation and the analysis of massive amounts of data.  Illinois Tech’s expertise in scientific computation lies in all College of Science Departments:  computer science, the life sciences, the physical sciences, and mathematics as well as in other Illinois Tech units.  Following a long term discussion, CISC was established in May of 2017 to leverage this expertise for greater impact.

Vision
CISC will be a national and international center of excellence in scientific computation underpinning and catalyzing multiple research and educational activities at Illinois Tech within the university, in Chicago, and beyond.
Mission
To intensify computationally-driven scholarship and education, across the College of Science, Illinois Tech as a whole, and in greater Chicagoland.  This may lead to major scientific advances not otherwise possible.
Goals
To fulfill its mission, CISC will initiate and promote programs to enhance research, education, and community engagement.  Our goals are the following:
•	Attract substantial external funding to support major research initiatives.  To increase the impact of our scientific computation research, we will form teams of experts comprised of multiple research groups inside and outside Illinois Tech.  On an annual basis one-year seed grants will be awarded to teams deemed to have the greatest potential for attracting major new external funding.  Regular seminars where research groups highlight their expertise and agendas will promote the matchmaking of these multidisciplinary teams.  Eventually, we expect to have a handful of externally funded major research programs at any time, with the particular themes depending our expertise and on the priorities of funding agencies.
•	Strengthen Illinois Tech’s research computing infrastructure.  We will partner with Illinois Tech’s Office of Technology Services (OTS) and sister institutions to provide access to the computing environments required for quality scientific computation research and education.  This includes promoting the efficient use and sharing of available computer resources beyond single-user machines, such as the new College of Science von Neumann cluster, GridIIT, the Open Science Grid, and XSEDE.  We will identify or sponsor training opportunities for scientific computation researchers who want to take advantage of high performance computing.  We will work with OTS and the university administration to acquire additional needed resources.
•	Develop a comprehensive scientific computation curriculum.  We will support Illinois Tech’s degree programs that have a scientific computation component and establish new multidisciplinary scientific computation programs.  We will consider development of an undergraduate major or minor in Scientific Computation as well as a Professional Science Masters program linking disciplinary computational knowledge with the business and entrepreneurial skills necessary in the tech start-up world.  We will also consider a PhD in Scientific Computation.  Our efforts will be directed towards both timely content and effective teaching methods.
•	Engage the community.  Organizations outside academia, including companies and non-profit entities need scientific computation to accomplish their missions. We will partner with these organizations through research collaborations, providing consulting, and placing our students for short-term internships and long term career opportunities.  We will make Illinois Tech a recognized source of expertise in scientific computation. 
Organization
The first Director of CISC is Fred Hickernell, Professor of Applied Mathematics.  Professor Hickernell’s expertise is computational mathematics and statistics, and his research has been funded by the National Science Foundation and the Department of Energy.  Eventually we intend to have an endowed chair, which will be used to recruit the next director. 
Initially. CISC will not have a formal membership.  Membership implies benefits and responsibilities.  It also implies inclusion and exclusion.  At the start, we want to encourage participation from a wide spectrum of faculty involved in scientific computation, including those outside the College of Science and outside of Illinois Tech.  In the future when the criteria defining membership become clear, we will have members.
The CISC website will highlight ongoing research and education at Illinois Tech.  It will also serve as a resource for computing environments, relevant conferences, and available training opportunities. 
Space
CISC will be housed in the Pritzker Science Center Office Suite 106.  This will include office space for the director, the assistant, visitors, and students.  There will also be a conference room and a common area.  The von Neumann cluster consisting of 32 nodes, each with 16 cores, is being housed in RE 214.  There is a long term need for Illinois Tech to identify and potentially renovate space to house the computer clusters acquired by faculty members and the Center.
Budget
The Center’s budget will be directed at accomplishing the goals outlined above.  The initial annual operating budget is $50K, to grow as funds become available. At full strength, CISC will require funding for both start-up and operating costs. It is hoped that these needs can be met through expendable gifts and/or endowment plus external grants and perhaps some part of the indirect cost recovery from these 
