\documentclass[12pt]{amsart}

\usepackage{booktabs, hyperref}
\usepackage[dvipsnames]{xcolor}


\textwidth 6.5 in
\textheight 9 in
\hoffset -0.8 in
\voffset -0.5 in

\newcommand{\FJHNote}[1]{{\textcolor{blue}{FJH: #1}}}
\newcommand{\DMNote}[1]{{\textcolor{green}{DM: #1}}}
\newcommand{\MCNote}[1]{{\textcolor{orange}{MC: #1}}}
\newcommand{\SCCNote}[1]{{\textcolor{magenta}{SCC: #1}}}


\begin{document}

\title[Center for Interdisciplinary Scientific Computation Strategic Plan]{Center for Interdisciplinary Scientific Computation \\ Strategic Plan}
\date{August 21, 2018}
\maketitle


Illinois Tech's mission, ``To provide distinctive and relevant education in an environment of scientific, technological, and professional knowledge creation and innovation," requires us to strengthen and leverage our expertise in computation as tool for discovery.  The Center for Interdisciplinary Scientific Computation (CISC) can play a key role. This strategic plan  summarizes our history, purpose, past and future activities, and proposed funding sources.

Computation, alongside theory and experiment, is a crucial tool in explaining phenomena, predicting what has not yet been observed, and optimizing performance.  Scientific computation encompasses both simulation science and data science. It draws upon multiple disciplines, including
\begin{itemize}
    \item Science and engineering domain knowledge to ensure that the problems posed are relevant and that the answers computed are significant,
    \item Computer science and engineering for novel algorithms, computational software, and hardware architectures that facilitate efficient and reliable computation, and
    \item Mathematical sciences for the rigorous framework and tools that justify computational procedures and inference from data.
\end{itemize}
At Illinois Tech scientific computation cannot be owned or led by a single department, but it must be fostered by an interdisciplinary center.

\subsection*{Background} Illinois Tech's expertise in scientific computation lies in all College of Science departments as well as in other Illinois Tech colleges, schools and departments.  Following a discussion spanning several years, CISC was established in May, 2017 to leverage our expertise for greater impact.  Fred Hickernell, professor of applied mathematics is the director, and David Minh, assistant professor of chemistry, is the associate director.  The CISC Advisory Board is comprised of 
\begin{itemize}
    \item Grant Bunker, Chair of the Department of Physics, 
    \item John Georgiadis, Chair of  the Department and R. A. Pritzker Professor of Biomedical Engineering,
    
    \item Ron Landis, 
Nambury S. Raju Professor of Psychology,

    \item Chun Liu, Chair of the Department of Applied Mathematics, and

    \item Xian-He Sun, Distinguished Professor of Computer Science.
\end{itemize}

\subsection*{Vision}
CISC will be a national and international center of excellence in scientific computation underpinning and catalyzing multiple research and educational activities at Illinois Tech within the university, in Chicago, and beyond.

\subsection*{Mission}
To intensify computationally-driven scholarship and education across the College of Science, Illinois Tech as a whole, and in greater Chicagoland.  This will lead to major scientific advances not otherwise possible.

\subsection*{Goals}
To fulfill its mission, CISC initiates and promotes programs to enhance research, education, and community engagement.  Our goals are the following:
\begin{itemize}
    \item Attract substantial external funding to support major \emph{research},
    
    \item Develop comprehensive and multi-pronged scientific computation \emph{education},

    \item Strengthen Illinois Tech's research computing \emph{infrastructure}, and
    
    \item Engage the \emph{community}.
    
\end{itemize}
We expect to see substantial progress within a three-year time horizon.

\subsection*{Research} CISC aims for at least a handful of externally funded major interdisciplinary scientific computation research programs at any time, with the particular themes depending our expertise and on the priorities of funding agencies.  A main weakness we have in competing for external funding for scientific computation research is our ignorance of what others in our own institution are doing and our lack of experience discussing problems across disciplinary boundaries.  CISC has and will take measures to overcome this weakness.

To promote interdisciplinary collaboration on computational projects, we instituted a series of \emph{lunchtime matchmaking seminars} in the fall of 2017.  Each seminar featured two twenty minute talks by two professors highlighting their computational research and the areas for potential collaboration.  Four applied mathematicians, two biologists, two biomedical engineers, a business faculty, a chemical and biological engineer, two chemists, two computer scientists,  a mechanical engineer, a physicist, and a psychologist spoke during the 2017--2018 academic year. 

The seminars were viewed enthusiastically in a survey conducted in the summer of 2018. Individuals commented that they ``enjoyed the interactions and opportunity to learn about research on campus'' and that the seminars were ``very helpful to start collaborations'', ``interesting for broadening awareness of current research'', and ``Gives a nice overview and lets people start talking to each other! Never discontinue this.''

We will continue this series, but we also want to facilitate the next step toward externally funded computational research by offering more substantial support to kickstart interdisciplinary collaborations.  We propose to offer \emph{summer research student stipends} to students supervised by faculty from more than one department.  These would be awarded based on the prospects of the summer work leading successful external funding of new interdisciplinary computational science research. 

In the fall of 2017, we invited interdisciplinary \emph{seed grant} proposals to fund scientific computation research in the 2018 calendar year leading to new external funding proposals.  Members of the Advisory Board reviewed the proposals and awarded a \$40K seed grant to Lulu Kang, Sonja Petrovi\'c, and Mahima Saxena. The funds were earmarked for teaching relief, data collection, and conference travel.  We plan to continue this seed grant competition on an annual basis.  Applicants are given latitude in proposing how use the funds.  For example, seed grants might be used for partial post-doctoral scholar support.  The primary criterion for selection is the greatest prospect of future external funding of new initiatives. Sonja Petrovi\'c says, ``The seed grant has allowed me to cross disciplinary boundaries, which is always time consuming at the start of any new collaboration.  Because of this grant, our team is able to dedicate time to a project that first requires a deep understanding of the problem in psychology, and then flexibility, patience, and innovation in applying statistical methods to the data from the psychological experiment.''

CISC has sponsored \emph{guest lectures} given by scholars leading major large scale computation projects. Our aim is to open the eyes of colleagues and students to what large scale scientific computation might be done.  Two of these guest lecturers were from nearby Department of Energy labs.  CISC hopes to strengthen our collaboration with these labs, which have access to advanced hardware and can provide valuable experience to our students.

Illinois Tech-hosted \emph{conferences} raise our visibility and help connect our faculty and students with experts from other institutions.  They can also generate new research initiatives.  We propose to provide modest financial support to those organizing scientific computation conferences here.


\subsection*{Education} Our scientific computation students are typically educated within their disciplinary silos, i.e., they are educated as computational biologists, computational chemists, computational mathematicians, computational physicists, or computer scientists.  Moreover, many students only gain experience with single CPU computation. We want to provide a richer experience for our students.

We want to break down silos.  Our aim in education is to
\begin{itemize}
    \item Place scientific computation students in research groups outside their disciplines so that they may learn how to think in new ways and work in interdisciplinary teams, some of which might be at government labs;
    
    \item Develop new scientific computation courses offered in various departments that can be taken by students from multiple disciplines, these courses will stress the use of high performance architectures, good professional practices of developing software for reproducible research, and scientific applications;
    
    \item Promote graduate theses incorporating advanced cyberinfrastructure; and
    
    \item Offer scientific computation research experiences for community college students and high school students.

\end{itemize}
Earlier this year CISC submitted a \emph{training grant} proposal to the NSF Office of Advanced Cyberinfrastructure that would fund these initiatives.  Although this proposal was recently declined, we will revise and re-submit in response to a future call.

We plan to continue to apply for training grants to support our aim of building a strong interdisciplinary scientific computation education for high school through PhD students.  Another possible source of funding are the Graduate Assistance in Areas of National Need (GAAN) grants. 

As an interdisciplinary center, CISC is well-situated to promote scientific computation education innovation at Illinois Tech.  We will partner with academic units to develop degree programs, minors, and certificates that reflect the cross-disciplinary perspective of scientific computation.  Our new scientific computation programs will prepare students to fill the increasing need for experts in computation for research, development, and operation in our modern economy.

Our survey feedback highlighted the desire to use more advanced computation resources.  A number of faculty and students have observed the limitations of their present computing environments.  They would benefit from on-site clusters, remote resources such as the Open Science Grid (\href{http://opensciencegrid.org}{\nolinkurl{opensciencegrid.org}}), and XSEDE (\href{https://www.xsede.org}{\nolinkurl{www.xsede.org}}), as well as new high performance languages and software libraries.  But the learning how to use these is a challenge.

CISC proposes to organize \emph{workshops} to help faculty and students take advantage of these advanced hardware and software resources.  The time scale of each workshop would be several days, and the topics would be based on demand. Workshop material would be made available online for future reference.  The organizer/instructor, either internal or external, would be paid a modest stipend.

\subsection*{Infrastructure} The Office of Technology Services (OTS) has turned its attention to supporting research computing in recent years.  CISC intends to partner with OTS and the Illinois Tech senior administration to support scientific computation research in ways that individual faculty or departments cannot do on their own.

Illinois Tech has established gridIIT, our piece of the Open Science Grid.  We are in dialogue with OTS on how to flatten the curve for using gridIIT by making templates and other training materials available. See also our plans for skills-based workshops in the previous section.

Several Illinois Tech faculty have recently purchased clusters but there is \emph{inadequate} space, power, and cooling to house them.  Because some of these clusters have configurations tailored to specific scientific problems, they cannot be replaced by computer time on the cloud.  CISC would like to lead the discussion on how to solve the problem.

The 256 node von Neumann cluster, purchased by the College of Science and managed by CISC, is on gridIIT and available for use campus wide.  The von Neumann cluster will need periodic updates.  At present, we would like to upgrade the communication among nodes to Infiniband.  There may be demand for a GPU-box.  In several years time, we will need to replace the compute nodes.  Thus, we intend to set aside funds for maintenance, upgrades, and replacement.

At the suggestion of faculty, CISC had begun compiling a database of licenses that we already hold for major software packages used in scientific computation (\href{http://bit.ly/2LrQC4T}{\nolinkurl{bit.ly/2LrQC4T}}),  such as ABAQUS, ADF, and Gaussian.  We hope that this will facilitate greater access to these packages.

\subsection*{Community} Illinois Tech is situated near 
\begin{itemize}
    \item Several research universities,
    
    \item Argonne and Fermilab, which are engaged in substantial high performance computing,
    
    \item Companies enabling high performance computing, e.g., NAG and AnyLogic,  and
    
    \item Companies that require heavy simulation or data analytics to thrive.
\end{itemize}
CISC wants to build stronger connections with this community so that we can offer them our intellectual resources, and we can take advantage of what they want to share with us.  We will invite them to campus, and we will visit them.  We will partner with them in our grant proposals and conference organization.  We will explore consulting opportunities.  We will connect them with our students looking for further study, internships, and careers in scientific computation.
As we strengthen our connections with the community, we will invite a few to join our Advisory Board, which presently consists only of internal members.

\subsection*{Membership} 
Up to now CISC has had no formal membership.  We will begin inviting faculty to join as \emph{CISC Affiliates}, starting with those who have participated in past CISC activities.  CISC Affiliates will be asked to indicate their experience in scientific computation and how they plan to be engaged in CISC, e.g. submitting grants associated with CISC, offering lectures or tutorials, or participating in curriculum development. The number of potential CISC Affiliates, i.e., faculty involved in scientific computation, is several dozen.  

\subsection*{Budget}
The initial CISC annual budget required to support our proposed activity is 
\begin{center}
    \begin{tabular}{p{0.57\textwidth}@{\qquad} r}
\textbf{Item} & \textbf{Annual Expense in \$K} \\
\toprule
    Seed grant  & 40 \\
    Summer research student stipends 3 @ \$7K each & 21 \\
    Cluster hardware upgrades & 30 \\
    Lunchtime matchmaking seminars & 2 \\
    Special lectures & 2 \\
    Training workshop instructor stipends 2 @ \$2.5K each & 5\\
    Student helper 30 weeks $\times$ 5 hours/week $\times$ \$15/hour & 7 \\
    Office expenses and other miscellany & 6 \\
    \bottomrule
    \emph{Total} & 113
\end{tabular}
\end{center}
This does not include teaching relief and administrative increment for the Director and Associate Director, which are presently supported by the College of Science.  As additional funds become available we would like to add the following:
\begin{center}
    \begin{tabular}{p{0.57\textwidth}@{\qquad} r}
\textbf{Item} & \textbf{Annual Expense in \$K} \\
\toprule
    Additional seed grant  & 40 \\
    Conference sponsorship & 10 \\
    \bottomrule
    \emph{Total} & 50
\end{tabular}
\end{center}

\subsection*{Financial Resources}
We aim to have stable income of at least \$170K per year within three years.
There are several potential sources of income to support CISC.  

One is an \emph{endowment}.  We were under the impression that donors with substantial means were willing to support CISC until spring 2017, when it became clear otherwise.  We are eager to work with Institutional Advancement to approach potential donors to build an endowment, whose interest would contribute to the annual budget.  We would also seek an endowed professorship for the next CISC director, also within three years.

Another potential source of income is the \emph{university's recurring budget}.  The College of Science supported CISC at a level of \$50K in the 2017-18 fiscal year, which was used to support the \$40K seed grant and various other CISC activities.

As highlighted under our goals, CISC is eager to partner with \emph{research} groups seeking external funding for major scientific computation projects.  We believe that CISC participation, with its infrastructure, will strengthen research proposals.  However, CISC's research scope is too broad for its activities in support of research to be externally funded.  We will compete for external \emph{training and education} grants, but these will not support much of the annual expenses outlined above.

We propose that a portion of the \emph{indirect costs} associated with all grants submitted in association with CISC be returned to CISC to support its annual expenses.  The advantages of this proposal are:
\begin{itemize}
    \item CISC is incentivized to support activities that lead to successful external proposals.
    
    \item Principal investigators (PIs) are incentivized to affiliate their proposals with CISC, knowing if successful, a portion of the indirect costs will be available to prime the pump for future proposals.  This assumes that the portion of indirect costs going to CISC does not affect the portion returned to the PI.
    
    \item The more successful CISC and our faculty are, the less CISC will rely on the recurrent university budget, and the more indirect costs will flow to the university.
\end{itemize}
This model is new to Illinois Tech and requires effort to be implemented.


\subsection*{Long Term Plans}
The CISC name and this strategic plan mention ``scientific computation'' and omit ``engineering''.  CISC was founded and initially resourced out of the College of Science.  However, from the beginning we have not limited our scope to science only, but have welcomed engineering and other disciplines.  Dean of Science Russell Betts appointed two members of our Advisory Board from outside science in consultation with their deans.  Faculty and students outside of science have participated in our activities.

Our plan is that within five years CISC to grow into, or become part of, a university-wide institute that would promote computational science, engineering, human sciences, and business.  This proposed institute would engage multiple disciplines.  We hope that Provost Kilpatrick will work with the deans and us to make that a reality.  Illinois Tech should be known as a leader in computation.








\end{document}



Is there some big picture perspective that we are missing?
Are there some activities that we should try to do or do differently?
Are there possible revenue sources that we should pursue?
What do you think about the proposed idea for membership?



