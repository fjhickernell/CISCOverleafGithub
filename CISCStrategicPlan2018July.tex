\documentclass{amsart}

\usepackage{booktabs}

\textwidth 6.5 in
\textheight 9 in
\hoffset -0.8 in
\voffset -0.5 in


\begin{document}
\title{Center for Interdisciplinary Scientific Computation \\ Strategic Plan}
\date{August 1, 2018}
\maketitle




To fulfill Illinois Tech's mission, ``To provide distinctive and relevant education in an environment of scientific, technological, and professional knowledge creation and innovation," we must strengthen and leverage our expertise in computation as tool for discovery.  The Center for Interdisciplinary Scientific Computation (CISC) is poised to play a key role. Here we summarize our history, purpose, past and future activities, and proposed funding sources.

Computation, alongside theory and experiment, is a crucial element in explaining phenomena, predicting what has not yet been observed, and optimizing performance.  Scientific computation draws upon multiple disciplines, including
\begin{itemize}
    \item Science and engineering domain knowledge to ensure that the problems posed are relevant and that the answers computed are significant,
    \item Computer science for novel algorithms, computational software, and hardware architectures that facilitate efficient and reliable computation, and
    \item Mathematical sciences for the framework justifying computational procedures and making inferences from data.
\end{itemize}
Thus, at Illinois Tech scientific computation is not owned or led by a single department, but is fostered through an interdisciplinary center.

Illinois Tech's expertise in scientific computation lies in all College of Science Departments as well as in other Illinois Tech colleges, schools and departments.  Following a long term discussion, CISC was established in May of 2017 to leverage this expertise for greater impact.  Fred Hickernell, Professor of Applied Mathematics was appointed as director.  Professor Hickernell's research in computational mathematics and statistics has been funded by the National Science Foundation and the Department of Energy.  In the spring of 2018, David Minh, Assistant Professor of Chemistry, was appointed Associate Director.  Professor Minh's research in ??? has been funded by ???.  The CISC Advisory Board is comprised of 
\begin{itemize}
    \item Professor Grant Bunker, Chair of the Department of Physics, 
    \item Professor John Georgiadis, Chair of  the Department and R. A. Pritzker Professor of Biomedical Engineering,
    
    \item Professor Ron Landis, Deputy Vice Provost for Research and Academic Affairs and
Nambury S. Raju Professor of Psychology,

    \item Professor Chun Liu, Chair of the Department of Applied Mathematics, and

    \item Professor Xian-He Sun, Distinguished Professor of Computer Science.
\end{itemize}

\textbf{Vision.}
CISC will be a national and international center of excellence in scientific computation underpinning and catalyzing multiple research and educational activities at Illinois Tech within the university, in Chicago, and beyond.

\textbf{Mission.}
To intensify computationally-driven scholarship and education, across the College of Science, Illinois Tech as a whole, and in greater Chicagoland.  This may lead to major scientific advances not otherwise possible.

\textbf{Goals.}
To fulfill its mission, CISC initiates and promotes programs to enhance research, education, and community engagement.  Our goals are the following:
\begin{itemize}
    \item Attract substantial external funding to support major research initiatives,
    
    \item Develop a comprehensive scientific computation curriculum,

    \item Strengthen Illinois Tech’s research computing infrastructure, and
    
    \item Engage the community.
    
\end{itemize}

To increase our competitive edge for external funding for interdisciplinary computational projects, we instituted a series of \emph{lunchtime matchmaking seminars} in the fall of 2017.  Each seminar featured two professors highlighting their computational research and the areas for potential collaboration.  Four applied mathematicians, two biologists, two biomedical engineers, a business faculty, a chemical and biological engineers, two chemists, two computer scientists,  a mechanical engineer, a physicist, and a psychologist spoke during the 2017--2018 academic year.  

While expecting to continue this series, we also want to facilitate the next step in collaboration by offering more substantial support to kickstart interdisciplinary collaborations.  We propose to offer \emph{summer research student stipends} to students supervised by faculty from more than one department.  These would be awarded based on the prospects of the summer work leading to a new interdisciplinary computational science proposal for external funding. 

Last fall we called for interdisciplinary \emph{seed grant} proposals to fund research in the 2018 calendar year leading to new external funding proposals in scientific computation.  Members of the Advisory Board reviewed the proposals and awarded a \$40K seed grant to Lulu Kang, Sonja Petrovi\'c, and Mahima Saxena. Part of the funding was for teaching relief and part for student support.  We would like to continue this on an annual basis.  We give the applicants latitude in proposing how use the funds.  For example, seed grants might also be used for partial post-doctoral scholar support.  The primary criterion for selection is the greatest prospect of leading to a successful proposal for external funding.

In addition to the 

\textbf{Resources.}

\textbf{Membership.} 


\textbf{Annual Budget.} 

\begin{tabular}{l@{\qquad} r}
\textbf{Item} & \textbf{Annual Expense in \$K} \\
\toprule
    Seed grant 2 @ \$40K each & 80 \\
    Summer research student stipends 3 @ \$7K each & 21 \\
    Lunchtime matchmaking seminars & 2 \\
    Student helper 30 weeks $\times$ 5 hours/week $\times$ \$15/hour & 7
\end{tabular}





\end{document}

Background
Modern science has been profoundly influenced by the development and use of computational methods for research and discovery.  This includes both large-scale simulation and the analysis of massive amounts of data.  Illinois Tech’s expertise in scientific computation lies in all College of Science Departments:  computer science, the life sciences, the physical sciences, and mathematics as well as in other Illinois Tech units.  Following a long term discussion, CISC was established in May of 2017 to leverage this expertise for greater impact.

Vision
CISC will be a national and international center of excellence in scientific computation underpinning and catalyzing multiple research and educational activities at Illinois Tech within the university, in Chicago, and beyond.
Mission
To intensify computationally-driven scholarship and education, across the College of Science, Illinois Tech as a whole, and in greater Chicagoland.  This may lead to major scientific advances not otherwise possible.
Goals
To fulfill its mission, CISC will initiate and promote programs to enhance research, education, and community engagement.  Our goals are the following:
•	Attract substantial external funding to support major research initiatives.  To increase the impact of our scientific computation research, we will form teams of experts comprised of multiple research groups inside and outside Illinois Tech.  On an annual basis one-year seed grants will be awarded to teams deemed to have the greatest potential for attracting major new external funding.  Regular seminars where research groups highlight their expertise and agendas will promote the matchmaking of these multidisciplinary teams.  Eventually, we expect to have a handful of externally funded major research programs at any time, with the particular themes depending our expertise and on the priorities of funding agencies.
•	Strengthen Illinois Tech’s research computing infrastructure.  We will partner with Illinois Tech’s Office of Technology Services (OTS) and sister institutions to provide access to the computing environments required for quality scientific computation research and education.  This includes promoting the efficient use and sharing of available computer resources beyond single-user machines, such as the new College of Science von Neumann cluster, GridIIT, the Open Science Grid, and XSEDE.  We will identify or sponsor training opportunities for scientific computation researchers who want to take advantage of high performance computing.  We will work with OTS and the university administration to acquire additional needed resources.
•	Develop a comprehensive scientific computation curriculum.  We will support Illinois Tech’s degree programs that have a scientific computation component and establish new multidisciplinary scientific computation programs.  We will consider development of an undergraduate major or minor in Scientific Computation as well as a Professional Science Masters program linking disciplinary computational knowledge with the business and entrepreneurial skills necessary in the tech start-up world.  We will also consider a PhD in Scientific Computation.  Our efforts will be directed towards both timely content and effective teaching methods.
•	Engage the community.  Organizations outside academia, including companies and non-profit entities need scientific computation to accomplish their missions. We will partner with these organizations through research collaborations, providing consulting, and placing our students for short-term internships and long term career opportunities.  We will make Illinois Tech a recognized source of expertise in scientific computation. 
Organization
The first Director of CISC is Fred Hickernell, Professor of Applied Mathematics.  Professor Hickernell’s expertise is computational mathematics and statistics, and his research has been funded by the National Science Foundation and the Department of Energy.  Eventually we intend to have an endowed chair, which will be used to recruit the next director. 
Initially. CISC will not have a formal membership.  Membership implies benefits and responsibilities.  It also implies inclusion and exclusion.  At the start, we want to encourage participation from a wide spectrum of faculty involved in scientific computation, including those outside the College of Science and outside of Illinois Tech.  In the future when the criteria defining membership become clear, we will have members.
The CISC website will highlight ongoing research and education at Illinois Tech.  It will also serve as a resource for computing environments, relevant conferences, and available training opportunities. 
Space
CISC will be housed in the Pritzker Science Center Office Suite 106.  This will include office space for the director, the assistant, visitors, and students.  There will also be a conference room and a common area.  The von Neumann cluster consisting of 32 nodes, each with 16 cores, is being housed in RE 214.  There is a long term need for Illinois Tech to identify and potentially renovate space to house the computer clusters acquired by faculty members and the Center.
Budget
The Center’s budget will be directed at accomplishing the goals outlined above.  The initial annual operating budget is $50K, to grow as funds become available. At full strength, CISC will require funding for both start-up and operating costs. It is hoped that these needs can be met through expendable gifts and/or endowment plus external grants and perhaps some part of the indirect cost recovery from these 
